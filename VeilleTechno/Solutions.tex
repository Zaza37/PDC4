\section{progicielle de B.I}

\subsection{Introduction}

\paragraph{} Nous venons de voir l’importance stratégique de la business intelligence de le monde de l’entreprise.

\paragraph{} C’est une donnée que les grands noms de l’informatique ont bien intégrée. En effet, le marché des logiciels de B.I. est principalement dominé par des solutions des grands noms de l’informatique (SAP, IBM, SAS, Oracle, Microsoft - soit un total de part de marché de 65\% pour les 5 premiers).

\paragraph{} C’est donc un domaine assez peu ouvert à la concurrence, bien que, nous le verrons, quelques outsiders ont réussi à s’imposer.
Des solutions libres ont également vu le jour dans ce domaine fermé et souvent propriétaire, mais sont bien malheureusement encore marginale et incomplète, cependant, ces solutions étant relativement récente, leur communauté se développant, il sera peut-être possible de voir dans le futur un accroissement de leur utilisation.

\paragraph{}Parmi l’ensemble des solutions différentes disponibles sur le marché, chacune possède des spécificités, des points forts et faiblesses propres.
\paragraph{} Les informations disponibles concernant les solutions de B.I. sont malheureusement assez restreintes, et se résument souvent aux informations commerciales fournies par les entreprises.

\paragraph{} Il est donc souvent difficiles de déterminer clairement quels sont les avantages et inconvénients principaux des solutions.  

\paragraph{} La partie suivante consiste en une tentative de présentation et comparaison des principales offres disponibles sur le marché. Nous présenterons les 6 offres leaders du marché du B.I., ainsi que 2 offres libres qu’il nous a paru intéressant d’évoquer.


\subsection{Solutions payantes}

\subsubsection{SAP Business Objects Enterprise Performance Management Solutions }

\paragraph{} Actuellement leader du marché, avec 21\% des parts de marché en 2010, SAP propose de nombreuses solutions adaptées aux différentes infrastructures d’entreprises allant des PMEs aux grosses multinationales.
\paragraph{} Cette solution offre aux clients une interface utilisateur simple et ergonomique. Particulièrement bien positionnée vis à vis des équipes de direction, la marque a su se développer notamment en proposant un système de Dashboards très complet et très apprécié.

\paragraph{} Cette solution possède cependant certains défauts la rendant parfois complexe à mettre en place. La marque propose effectivement diverses solutions ERP très utiles pour segmenter les secteurs stratégiques de l’entreprise mais entraînant des difficultés lors de la mise en place des Data Warehouses indispensables pour l’exploitation d’une solution Business intelligence.

\subsubsection{IBM Cognos Business intelligence and financial performance management}

\paragraph{} IBM, bien que ne possédant pas la plus grosse part de marché pour les solutions de Business Intelligence, est reconnu comme la solution la plus complète et est en plein déploiement stratégique. Sa place de second sur le marché de la Business Intelligence (avec 14.8\% des parts de marché en 2010), est sans doute due à une arrivée tardive sur le marché, suite au rachat de Cognos Inc. en 2008 ainsi que de nombreuses autres solutions depuis.
	
\paragraph{} C’est l’une des rares plates-formes multi-OS, et face à la densité des offres concurrentielles à forte valeur ajoutée, c’est un plus produit permettant un démarquage stratégique essentiel pour la solution. Elle est de plus la seule marque à proposer  une solution sur zOS, le système d’exploitation des mainframes , très présentes dans les établissements bancaires.
	
\paragraph{} La principale force de cette solution face à la concurrence est la qualité technique et l’ergonomie générale qu’elle propose. Elle possède notamment un module d’analyse prédicitve très recherchée par les entreprises modernes afin de prendre des décisions stratégiques.


\subsubsection{Business Intelligence SAS Institute}


\paragraph{} La solution de Business Intelligence de SAS est actuellement troisième sur le marché du B.I. (avec 11\% des parts de marché).

\paragraph{} C’est une solution se démarquant de la concurrence principalement par ses modules de prédiction (qu’ils sont les seuls à intégrer, selon la fiche commerciale de leur site).
C’est aussi une solution multi plates-formes, intégrée et évolution.

\paragraph{} Les principaux défauts relevés par les utilisateurs sont la nécessité en ressources de la solution, ainsi que le haut niveau technique requis pour son utilisation, empêchant l’utilisateur lambda d’en faire une utilisation aisée et intuitive.

\subsubsection{Oracle, Hyperion Solution}

\paragraph{} Quatrième dans le marché du B.I. (avec 9\% des parts de marché en 2010), la solution d’Oracle est une solution assez mitigée.

\paragraph{} C’est une solution avec de nombreux points positifs. Son moteur OLAP est en effet reconnu pour ses bonnes performances, et les dashboards interactifs que présentent la solution sont également appréciés des utilisateurs. C’est un produit supportant le mode déconnecté. Enfin, la solution présente une excellente intégration avec Microsoft Office, la suite bureautique la plus utilisée dans la monde de l’entreprise.

\paragraph{} Cependant, la solution possède ses défauts. Tout d’abord, ce produit ne déroge pas à l’image de la “Big Red Stack” d’Oracle, à savoir qu’Oracle impose l’utilisation de l’ensemble de ses solutions sur l’ensemble de la chaîne de traitement de l’information pour une intégration correcte de son produit - cela va du stockage, avec les serveurs, jusqu’aux applications, avec Hyperion, en passant pas des VM, l’OS, les bases de données, les middleware -.

\paragraph{} Enfin, c’est une solution assez technique et difficile à prendre en main, qu’il n’est pas possible de laisser dans n’importe quelle main donc.

\subsubsection{Microsoft SharePoint Server}

\paragraph{} En cinquième position sur le marché du B.I. (9\% des parts de marché en 2010), la solution de Microsoft est aux coudes à coudes avec la solution d’Oracle.

\paragraph{} C’est une solution considérée comme étant bonne, robuste, et ergonomique.

\paragraph{} Elle intègre logiquement une excellent intégration de Microsoft Office. Ses solutions de reporting et de dashboards, sans être les meilleures du marché, sont tout de même considérées comme étant plutôt perfomrantes.

\paragraph{} La solution Microsoft permet de faire du B.I. collaboratif.

\paragraph{} Le principal inconvénient de la solution étant la nécessité de l’installer sur chaque poste, et les mises à jour, plutôt fréquentes, sont à l’usage très contraignantes et lourdes pour l’utilisateurs.

\subsection{Solutions Open-Source}

\subsubsection{Pentaho}

\paragraph{} Cette solution est open-source, partiellement gratuite. Elle a la particularité d’être très intuitive et ainsi de permettre aux utilisateurs une prise en main rapide. 

\paragraph{} Elle offre une solution évolutive et flexible. Il lui manque cependant quelques composants pour concurrencer réellement les principaux acteurs du marché.

\subsubsection{Jaspersoft}

\paragraph{} Solution open-source écrite en java, elle propose de nombeuses fonctionnalités, elle a la particularité d’être multi-plateforme. Elle est également l’une des seules solutions existantes à proposer des interfaces mobiles.

\paragraph{} Tout comme la solution précédente, elle manque encore de maturité pour être une réelle alternative aux principales offres du marché. Sa gratuité complète permet cependant à des PMEs de disposer d’un outil fonctionnel de Business Intelligence qu’elles n’auraient pu financer.
