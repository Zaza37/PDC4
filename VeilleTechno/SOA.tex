\section{Solutions orientées SOA}
La logique même du SOA est de fournir à un SI la capacité à se développer en adjoignant une logique de composants au SI. Ainsi, il est cohérent d’envisager la piste du déploiement d’une solution de business intelligence dans une démarche SOA.\\
Les composants clés de ces systèmes sont :
\begin{itemize}
\item[•]EAI : Enterprise Application Integration  : Système de communication entre les différentes applications opérationelles, visant à assurer la cohérence entre le DW et les différentes bases de données opérationelles, déployées chez l’utilisateur.
\item[•]Master Data Management :  (Optionnel, mais potentiellement critique à long terme )  Système d’optimisaiton de la qualité basé sur la création d’un sous enssemble de données de références au plus justes possibles, permettant à long terme de garantir la qualité des données et une cohérence dans l’exploitation des données transactionnelles
\item[•]ETL : Extract, Transform, Load : Directement Raccordé à l’EAI, il s’agit d’un utilitiare la encore basées sur les Métadonnées, qui permet d’assurer l’extraction, l’uniformisation et la structuration des données issurées des BDD opérationelles, et leur chargement  / intégration au sein du Data Warehouse et des datamarts.
\item[•] DataWarehouse  : Schéma de BDD central, servant de base d’exploitation aux outils de  BI.
\item[•] Solutions de BI intégrées : Présentation, Reporting et exploitation : 
\end{itemize}
Principaux acteurs du marché :
\begin{itemize}
\item[•]Oracle
\item[•]IBM
\end{itemize}  