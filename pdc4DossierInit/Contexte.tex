\titlecolor{Objet et contexte du projet}
\subsection{Objet du projet}
\paragraph{} Ce projet vise à concevoir plusieurs scénarios permettant d'actualiser les architectures applicative et technique en place de la société Péchiney Electrométallurgie (PEM). Le Comité de Pilotage devra avoir les éléments permettant d'évaluer, comparer et choisir l'un de ces scénarios.
Dans le cadre de cette étude il faudra compléter les informations qui ont été recueillis lors des phases antérieures du schéma directeur en particulier les phases d'analyse de l'existant et d'analyse des besoins fonctionnels. Il sera nécessaire de faire une veille technologique des solutions informatique du marché. 
\subsection{Contexte du projet}
L'étude s'inscrit dans le cadre du projet d'élaboration du schéma directeur des systèmes d'information que la société PEM a engagé récemment pour les cinq prochaines années. La société Pecheney Electrométallurgie a réalisé un premier schéma directeur pour l'informatique dans les années 80 qui a conduit à une rénovation importante de son système d'information. L'architecture actuelle est donc en place depuis plusieurs années, elle couvre les principaux besoins du SI opérationnel de l'entreprise.\\ Les principaux objectifs du nouveau plan directeur sont dans ce contexte les suivants :
\begin{itemize}
\item[•]Maitriser les couts
\item[•]Augmenter sa réactivité vis à vis des fournisseurs
\item[•]Améliorer le pilotage de l'entreprise
\item[•]Améliorer la capacité du système d'information
\item[•]Garantir la pérennité du système informatique
\end{itemize}