\titlecolor{Organisation et ressources}
\subsection{Ressources}
L'équipe se compose de 4 personnes : Elsia Abidh, qui tiendra le rôle de chef de projet, et Adrien Brochot, Julien Levesy, Armand Rossius qui composeront le groupe d'étude chargé d'élaborer les scénarios.
\subsection{Outils de communication utilisés}
En dehors de réunions et revues, un certain nombre d'outils seront utilisés pour assurer la communication au sein de l'équipe.

\subsubsection{Mails}
Les mails seront utilisés principalement pour fixer des réunions et transmettre des documents entre les membres de l'équipe. Les discussions et débats devront quant à eux être réalisés en face à face. Le chef de projet pourra également communiquer par mail pour demander des corrections sur des résultats, rappeler des objectifs, des dates limites ou des tâches à venir quand le besoin s'en fera sentir trop loin d'une séance de travail en commun, ou sur demande d'un membre de l'équipe.

\subsubsection{Dossier Google Docs}
Un dossier partagé Google Docs a été créé. Il est accessible par tous les membres de l'équipe, et sert à la réflexion et à la transmission de documents. Les documents qu'il contient sont des brouillons ou des notes de réflexion. On ne pourra pas considérer les documents contenus dans ce dossier comme finaux, au moins dans la forme.

\subsection{Outils de travail}
La mise en page finale des documents, une fois l'équipe d'accord sur le fond effectué grâce à Latex. 
Tous les documents devront utiliser le même style, défini par
le chef de projet (en l'absence de Responsable Qualité sur ce projet, compte tenu de la petite taille de l'équipe). La présentation orale sera faite sous Keynote, un logiciel sous mac.