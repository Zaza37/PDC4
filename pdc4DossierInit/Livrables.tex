\titlecolor{Livrables}
\subsection{Livrables de gestion de projet}
\subsubsection{Dossier d'initialisation}
Le dossier d'initialisation présente l'objet du projet. Ce dossier constitue le fondement de l'organisation de l'équipe. Il définit les rôles et responsabilités de chacun. Il précise les différentes tâches et livrables du projet avec leur description. Il contient le planning et le plan de charge qui détaille le temps prévu pour chaque intervenant sur chaque tâche.
Bien que le plan de charge ne soit jamais respecté exactement, il permet d'identifier les tâches les plus longues et de suivre la consommation du temps de chaque intervenant par sa mise à jour tout au long du projet. Voici le plan du dossier d'initialisation :
\begin{itemize}
\item[•]Objet du projet
\item[•]Livrables attendus
\item[•]Organisation et ressources
\item[•]Activités, tâches et planification
\item[•]Procédures de gestion de projet
\item[•]Analyse des risques
\end{itemize}

\subsection{Dossier de production}
\subsubsection{Document de synthèse}
Il contiendra les informations nécessaire pour présenter au Comité de Pilotage les scénarios retenus : architectures cibles et plans de mise en œuvre. Il contiendra également les informations permettant l'évaluation, la comparaison et le choix d'un scénario par le Comité.
\subsubsection{Document d'annexes}
Ce document apportera les éclairages de la synthèse. 
\subsubsection{Document de synthèse de la veille technologique}
Ce document présentera une synthèse de la veille des solutions d'aide au pilotage décisionnel.
\begin{itemize}
\item[•]Définition des solutions de pilotage décisionnel
\item[•]Présentation des solutions existantes 
\end{itemize}
