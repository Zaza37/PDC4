\section{Critères}
\subsection{INTRODUCTION}

\paragraph{} La mise en place d'une nouvelle architecture de Système d'Informations est une opération particulièrement difficile à aborder. La nouvelle architecture doit répondre aux anciens besoins de l'entreprise tout en étant capable de s'adapter aux futures décisions stratégiques et évolutions de l'entreprise. La décision de mise en place d'une architecture impactera tous les niveaux de l'entreprise et doit être justifiée par un besoin et des améliorations significatives apportées par la nouvelle architecture. Notre objectif est de vous permettre de choisir parmi les solutions envisageables. Pour ce faire, nous avons mis en place une procédure d'évaluation complète et objective de chaque solution vous permettant de les noter et de dégager plus simplement la solution correspondant le mieux aux attentes de votre entreprise. Ces critères sont communs à tous les scénarios d'évolution proposés et permettront d'évaluer d'autres scénarios si besoin. Chaque scénario représentera un investissement conséquent et plusieurs années seront probablement nécessaires pour le remplacer. L'évolution du SI d'une entreprise a cependant pour objectif final de permettre une meilleure gestion des coûts limitant les dépenses inutiles et fournissant ainsi un bon Retour sur Investissement.

\subsection{MÉTHODE D'ÉVALUATION DES CRITÈRES}


\paragraph{} Afin d'aider dans le choix de l'orientation de la solution à prendre, il est nécessaire d'évaluer le plus objectivement possible chaque solution.
Il a donc été décidé d'établir une liste de critères la plus exhaustive et la plus objective possible.
Ces critères ont pour but d'évaluer la solution en lui attribuant une note. Plus précisément, à chaque critère correspond un poids et une note compris entre 0 et 5. Le poids du critère précise l'importance du critère dans le système d'évaluation (1 représente un critère peu important, tandis que 5 représente un critère essentiel), tandis que sa note indique la performance de la solution par rapport à ce critère (0 représente une performance médiocre en comparaison des autres solutions, et 5 représente la meilleur solution parmis toutes les autres).

\paragraph{} En multipliant la note de chaque critère par son poids, et en additionnant chaque critère, on obtient donc une note finale pour chaque solution.

\paragraph{} La difficulté de l'évaluation réside principalement dans la difficulté a évaluer objectivement un critère.

\paragraph{} Il a donc été décidé d'attribuer à chaque critère une mesure chiffrable et objective (un temps en heures, un coût en euros, etc...).

\paragraph{} A partir de cette mesure chiffrable, il est possible de déterminer deux types de critères :
les "critères positifs" : plus la valeur est haute, plus la note est importante (ex : le bénéfice)
les "critères négatifs" : plus la valeur est haute, plus la note est faible (ex : le coût)


\paragraph{} A partir de la valeur de chaque critère, il faut donc ensuite obtenir une note.
Deux méthodes ont été retenus pour définir une note pour chaque critère :
\begin{itemize}
\item la méthode "de proportionnalité", utilisée pour les "critères positifs"
\item la méthode "de proportionnalité inverse", utilisée pour les "critères négatifs"
\end{itemize}


\paragraph{} Détail de la méthode "de proportionnalité" :
\paragraph{} On attribut la note 5 à la solution ayant la valeur la plus élevée. On attribut ensuite à chaque autre solution la note définie par la formule suivante : 5* Valeur de la solutionValeur de la meilleure solution
\paragraph{} On constate donc que la meilleure solution obtiendra tout le temps (pour le critère en question) la note 5 et que les autres solutions auront une note comprise entre 0 et 5.

\paragraph{} Détail de la méthode "de proportionnalité inverse" :
\paragraph{} On attribut la note 0 à la solution ayant la valeur la plus élevée. On attribut ensuite à chaque autre solution la note définie par la formule suivante : 5 -(5* Valeur de la solutionValeur de la meilleure solution)
\paragraph{} On constate donc que la pire solution obtiendra tout le temps (pour le critère en question) la note 0 et que les autres solutions auront une note comprise entre 0 et 5.

\subsection{LISTE DES CRITÈRES}

\paragraph{} Nous présentons ici la liste des critères retenus pour l'évaluation des solutions. Ces critères sont répartis dans quatre catégories principales :

\begin{itemize}
\item Rentabilité
\item Déploiement
\item Satisfaction des objectifs
\item Autres
\end{itemize}

\paragraph{} Nous indiquons quelle méthode d'évaluation sera utilisée pour le critère ("proportionnalité" ou "proportionnalité inverse"), et nous précision également comment la mesure de chaque critère est calculée (si jamais cette mesure n'est pas évidente, comme pour le coût par exemple).

\subsubsection{RENTABILITÉ}

\begin{description}
\item[Coût :] Nous appliquerons la méthode de "proportionnalité inverse".
\item[Durée avant le ROI :] Nous appliquerons la méthode de "proportionnalité inverse". Nous appliquerons une méthode identique à celle utilisée pour le coût de la solution.
\item[Bénéfices par an (maitriser les coûts) :] Nous appliquerons la méthode de "proportionnalité".
\item[Rapport Bénéfices par an sur Coût total :] Nous appliquerons la méthode de "proportionnalité".

\end{description}

\subsubsection{DÉPLOIEMENT}

\begin{description}
\item[Durée d'implémentation :] Nous appliquerons la méthode de "proportionnalité inverse".
\item[Durée de la formation des employés (en jours) :] Nous appliquerons la méthode de "proportionnalité inverse".
\item[Impact des changements apportés par la solution :] On applique la méthode de "proportionnalité inverse" sur le nombre de postes à changer lors de la mise en place de la solution.
\item[Impact des modifications sur les utilisateurs pendant le déploiement :] Temps total d'indisponibilité des postes utilisateurs durant la mise en place. Nous utiliserons la méthode de "proportionnalité inverse".

\end{description}

\subsubsection{AUTRES}
\begin{description}
\item[Positionnement de la solution en terme d'écologie :] On applique la méthode de "proportionnalité inverse" sur l'empreinte écologique de la solution.

\end{description}

\subsubsection{Satisfaction des Objectifs}

\begin{description}
\item[Maîtriser les coûts (cf bénéfices) :]
\item[Augmenter la réactivité vis à vis des fournisseurs (réactivité calculée en jours) :] Nous appliquerons la méthode de "proportionnalité inverse".
\item[Augmenter les services auprès des clients :] On applique la méthode de "proportionnalité" sur le pourcentage de compatibilité de notre solution avec les standards du marché. 
\item[Améliorer le pilotage de l'entreprise :] On applique la méthode de proportionnalité inverse sur la fréquence de mise à jour (en heures) des outils de pilotage.
\item[Améliorer la capacité d'évolution du système informatique :] Coût moyen d'évolution de la solution lors d'un changement organisationnel ou stratégique de l'entreprise
\item[Garantir la pérennité du système informatique :] On applique la méthode de "proportionnalité" sur le pourcentage "d'uptime" annoncé par le constructeur de la solution.
\end{description}


\subsubsection{Évaluation des Solutions}

\paragraph{} Les solutions que nous avons sélectionnées vont maintenant être évaluées selon les critères décrits ci-dessus afin de déterminer laquelle sera la plus appropriées pour votre entreprise.

\definecolor{EnTete}{gray}{0.90}
\definecolor{Total}{gray}{0.75}
\definecolor{Activite}{gray}{0.80}

\begin{tabular}{|c|c|c|c|} 
\hline
\rowcolor{EnTete} Critère & poids & Solution 1 - ERP & Solution 2 - SOA \\ \hline
\rowcolor{Activite} RENTABILITE &  &  & \\ \hline
Coût & 2 & 0 & 4 \\ \hline
Durée avant le ROI & 3 & 0 & 3 \\ \hline
Rapport Bénéfices/Coût & 2 &  &  \\ \hline
Bénéfices / an & 5 &  &  \\ \hline
 &  &  &  \\ \hline
\rowcolor{Activite} DEPLOIEMENT &  &  &  \\ \hline
Durée d'implémentation & 3 & 0 & 2,5  \\ \hline
Formation des employés requise & 3 & 0 & 2 \\ \hline
Impact des changements apportés par la solution & 2 & 3 & 0 \\ \hline
Impact des modifications sur les &  &  &  \\
utilisateurs pendant le déploiement & 1 & 0 & 2 \\ \hline
 &  &  &  \\ \hline
\rowcolor{Activite} AUTRES &  &  &  \\ \hline
Positionnement de la solution en termes d'écologie & 1 & 0 & 1 \\ \hline
 &  &  &  \\ \hline
\rowcolor{Activite} SATISFACTION DES OBJECTIFS &  &  & \\ \hline 
Maitriser les coûts (cf bénéfices) &  &  & \\ \hline
Augmenter la réactivité vis à vis des fournisseurs & 4 & 0 & 2 \\ \hline
 Augmenter les services auprès des clients & 2 & 5 & 2 \\ \hline
Améliorer le pilotage de l’entreprise & 3 & 5 & 3 \\ \hline
Améliorer la capacité d’évolution du système  &  &  &  \\
informatique & 1 & 0 & 3 \\ \hline 
Garantir la pérennité du système informatique & 3 & 5 & 4 \\ \hline
 &  &  &  \\ \hline
\rowcolor{Total}TOTAL & 35 & 74.6 & 94.5 \\ \hline

\end{tabular}