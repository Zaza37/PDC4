\section{Etude des solutions d'architecture}

\subsection{Scénario d'Urbanisation du SI}

\subsubsection{Présentation de la solution}
\paragraph{Architecture Logicielle}

\paragraph{Architecture matérielle}

\subsubsection{Ressources nécessaires}

\subsubsection{Plan de déploiement}

\subsubsection{Gestion du changement}

\subsubsection{Conclusion}

\subsection{Scénario de déploiement d'un ERP}

\subsubsection{Présentation de la solution}
L’intégration de logiciels d’ERP (planification des ressources d’entreprise) consiste à intégrer des systèmes d’ERP propriétaires, anciens et indépendants, les uns aux autres ou avec d’autres ressources d’informations de l’entreprise, afin de répondre à divers besoins de B2B. Les systèmes d’ERP tels que ceux de SAP, Peoplesoft, Oracle et Lawson ont été conçus pour intégrer étroitement les processus d’une entreprise. Initialement, il s’agissait de la planification, de la fabrication et des ventes, mais les logiciels d’ERP plus récents couvrent également le marketing, le contrôle d’inventaire, le suivi des commandes, le service aux clients, les finances et les ressources humaines.
Propose des applications spécialisées pour la gestion des laboratoires (LIMS), de la relation client (CRM), de la maintenance (GMAO)
\paragraph{Architecture Logicielle}
soft hébergé quelque part
pas d'installation sur machine
acces via site web
Permet 
\paragraph{Architecture matérielle}
Une base de donnée unique répliquée sur chaque sites, ...
\subsubsection{Ressources nécessaires}
Pour le déploiement d'une tel solution on peut envisager deux cas : soit on embauche du personnel dédié à l'intégration d'ERP soit on sous traite et on embauche une société spécialisé dans le domaine.
\subsubsection{Plan de déploiement}
1]Analyser le flux d'information de l'entreprise
\subsubsection{Gestion du changement}

\subsubsection{Conclusion}
Avantages :
\begin{itemize}
\item[.]Optimisation des processus de gestion;
\item[.]Cohérence et homogénéité des informations (un seul fichier articles, un seul fichier clients, etc.) ;
\item[.]Intégrité et unicité du Système d'information ;
\item[.]Partage du même système d’information facilitant la communication interne et externe ;
\item[.]Minimisation des coûts : pas d’interface entre les modules, synchronisation des traitements, maintenance orrective simplifiée car assurée directement par l'éditeur et non plus par le service informatique de l'entreprise (celui-ci garde néanmoins sous sa responsabilité la maintenance évolutive : amélioration des fonctionnalités, évolution des règles de gestion, etc.) ;
\item[.]Globalisation de la formation (même logique, même ergonomie) ;
\item[.]Diminution du nombre de salariés ayant pour mission principale la saisie comptable (aide-comptable);
\item[.]Maîtrise des coûts, des délais de mise en oeuvre, de déploiement ;\\
\end{itemize}
La mise en oeuvre d'un ERP/PGI dans une entreprise est associée à une révision en profondeur de l'organisation des tâches et à une optimisation et standardisation des processus, en s'appuyant sur le « cadre normatif » de l'ERP/PGI.\\

Inconvénients :

\begin{itemize}
\item[.]Mise en œuvre pouvant être complexe si le périmètre est mal déterminé ou trop mouvant ou le projet mal piloté ;
\item[.]oût élevé de 300 000 € minimum pour un progiciel fiable et de qualité, mais pouvant rapidement monter beaucoup plus haut, en fonction de l'industrie et de la complexité du projet. L'option fonctionnellement riche des solutions de logiciels libres si elle réduit les coûts de licence, ne supprime pas les coûts d'accompagnement et de formation ;
\item[.]Périmètre fonctionnel souvent plus large que les besoins de l'organisation ou de l'entreprise (le progiciel est parfois sous-utilisé) ;
\item[.]Périmètre fonctionnel pouvant ne pas couvrir l'ensemble des besoins4 ;
    lourdeur et rigidité de mise en œuvre ;
    difficultés d'appropriation par le personnel de l'entreprise ;
    nécessité d'une bonne connaissance des processus de l'entreprise (par exemple, une petite commande et une grosse commande nécessitent deux processus différents : il est important de savoir pourquoi, de savoir décrire les différences entre ces deux processus de façon à bien les paramétrer et à adapter le fonctionnement standard du PGI/ERP aux besoins de l'entreprise) ;
    nécessité parfois d'adapter certains processus de l'organisation ou de l'entreprise au progiciel ;
    nécessité d'une maintenance continue ;
    captivité vis-à-vis de l'éditeur : le choix d'une solution est souvent structurant pour l'entreprise et un changement de PGI peut être extrêmement lourd à gérer.


\end{itemize}
